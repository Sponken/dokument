\subsection{\DIGITFULL}
\subsubsection{Sammansättning}
\DIGIT{} består av ordförande, kassör samt 1-6 övriga ledamöter.

\subsubsection{Inval}
\DIGIT{}s medlemmar skall varje år väljas på det första ordinarie vårmötet.

\subsubsection{Det åligger \DIGIT}
\begin{att}
	\item underhålla och utveckla system för sektionen som tillsammans uppfyller följande behov:
	\begin{itemize}
		\item En webbplats som kan presentera nyheter och sponsorer/partner.
		\item Ett system som sektionsmötet kan använda för att genomföra sluten votering
		\item Ett system som ser till att alla organ inom sektionen får tillgång till en epostadress som kan nyttjas av organet
		\item Ett system som låter gemene IT-teknolog eller organ på sektionen boka delar av, eller hela sektionslokalen
		\item Regelbundna säkerhetskopior på data från de system och tjänster som tillgodoser kraven i denna listan
	\end{itemize}
	\item tillsammans med sektionens rustmästeri underhålla och utveckla de digitala system som finns i sektionslokalen.
	\item tillhandaha ett medlemsregister med information om vilka som för tillfället är invalda i sektionens kommittéer, föreningar och andra instanser.
	\item uppmuntra till en god kodkultur på sektionen
\end{att}

\subsubsection{Det åligger \DIGIT{}s ordförande}
\begin{att}
	\item leda \DIGIT{}s verksamhet.
	\item handha \DIGIT{}s handlingar.
\end{att}

\subsubsection{Det åligger \DIGIT{}s kassör}
\begin{att}
	\item handha \DIGIT{}s ekonomi.
\end{att}

\subsubsection{Mandatperiod}
Invalda till \DIGIT{} betraktas som en del av \DIGIT{} och sektionen från tillträdet, måndagen efter invalet, till avträdet, 1:a maj ett år efter tillträdet.

Det yttersta ansvaret för \DIGIT{}s verksamhet och ekonomi övergår till de nyinvalda den 1:a maj.


